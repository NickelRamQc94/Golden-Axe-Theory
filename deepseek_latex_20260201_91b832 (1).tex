Théorème (Nickel, 2026) :
∀ écoulement turbulent stationnaire ∃ Π_N ∈ ℝ tel que :
  
1. Π_N = C × (ATI/TCF)^α × (1 + ε·F(MENdS, MENeS))
2. α ≈ 0.75 ± 0.03 (universel)
3. C = C(géométrie) (spécifique)

Preuve esquissée :
Soit L_η l'échelle de Kolmogorov, L_int l'échelle intégrale.
Par analyse dimensionnelle :

Π_N = ⟨ω²⟩^(1/2) / ⟨|∇u|²⟩^(1/4) × (L_η / L_int)^(1/3)

En utilisant le théorème Π de Buckingham,
on obtient la forme d'échelle ci-dessus.